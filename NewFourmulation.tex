\documentclass[onecolumn,11pt,journal, compsoc]{IEEEtran}

\usepackage{anysize}
\marginsize{1.25in}{1.25in}{1.0in}{1.0in}
\usepackage{algorithm,algorithmic}
\usepackage{multirow}
\usepackage{url}
\usepackage{graphicx} 
\usepackage{subfigure}
\usepackage{enumitem}
\usepackage{ctable}
\usepackage{amssymb,amsmath}
\usepackage{cite}
\usepackage{subfigure}
\usepackage{wrapfig}
\usepackage{epstopdf,epsfig}
\usepackage{amsmath}

\newtheorem{theorem}{Theorem}[section]
\newtheorem{lemma}[theorem]{Lemma}
\newtheorem{proposition}[theorem]{Proposition}
\newtheorem{corollary}[theorem]{Corollary}

\newenvironment{proof}[1][Proof]{\begin{trivlist}
		\item[\hskip \labelsep {\bfseries #1}]}{\end{trivlist}}
\newenvironment{definition}[1][Definition]{\begin{trivlist}
		\item[\hskip \labelsep {\bfseries #1}]}{\end{trivlist}}
\newenvironment{example}[1][Example]{\begin{trivlist}
		\item[\hskip \labelsep {\bfseries #1}]}{\end{trivlist}}
\newenvironment{remark}[1][Remark]{\begin{trivlist}
		\item[\hskip \labelsep {\bfseries #1}]}{\end{trivlist}}

\newcommand{\qed}{\nobreak \ifvmode \relax \else
	\ifdim\lastskip<1.5em \hskip-\lastskip
	\hskip1.5em plus0em minus0.5em \fi \nobreak
	\vrule height0.75em width0.5em depth0.25em\fi}

\newcommand*{\QEDB}{\hfill\ensuremath{\square}}%


\begin{document}

	Given a set of files $\mathcal F = \{F_1, ..., F_n\}$ with equal size $s$, where $s$ is the number of packets in $F_i$, and we generate $s_g = (1+\theta)s$ packets where $\theta \in [0, 1] $ . We represent each file $F_i \in \mathcal F$ as a node $F_i=<s_i, d_i, Generate_i>$ where $s_i, d_i$ denote the source and destination for that file and $Generate_i$ is the time when the file is generated. For each node $F_i=<s_i, d_i, Generate_i>$, we add a virtual source node $sv_i$ having an edge with capacity $s_g$ into file node $F_i$. And we add a virtual destination node $dv_i$ having an incoming edge of infinite capacity from all the connection nodes which contain $d_i$ as destinations. In this way we can model this problem as multi-commodity flow problem with $n$ different kinds of commodities with $n$ source and destination pairs of $(sv_i, dv_i)$. Assume that all commodity can be fully fractionally delivered alone, which means each commodity can always be delivered in the form of splittable packets in the absence of other commodities.
	
	
	\textbf{Formulation 1}: Maximize the total number of files that can be successfully decoded.
	
	Let $\lambda \in [0, 1], \lambda << \theta$ be the threshold percentage for each file to be successfully decoded, which means if $s_r = (1+\lambda)s$ packets can be delivered to destination, file $F_i$ can be treated as successfully delivered. We use a variable $f_i$ to represent whether or not $F_i$ is successfully delivered. For the sake of simplicity, we rescale each file and edge capacity in the network by $s_r$ so that this problem thus becomes an unit flow multi-commodity problem.
	
	\begin{equation}
		\begin{aligned}
			\text{Input: Directed Connection Graph } &G(V, E), \text{File set } \mathcal{F} \text{ with } n \text{ files} \\
			\text{Source-sink~pair }&(sv_i, dv_i) \text{ and threshold }\lambda, ~\forall F_i \in \mathcal{F} \\
			\text{ Capacity }&c(u, v) ~~~~\forall (u, v)\in E \\
			\text{Variable: Flow } &f_{i, (u, v)} ~~~~~~\forall F_i \in \mathcal{F}, (u, v) \in E \\
			&f_i  ~~~~~~~~~~~~~\forall F_i \in \mathcal{F} \\
			\text{Maximize } &\sum_{i=1}^{n}f_i \\ 
			\text{Subject to } &\sum_{(sv_i,v) \in E}f_{i, (sv_i,v)} = f_i,  ~~~~~~~~~~~~\forall F_i \in \mathcal{F} \\
			&\sum_{(u, v) \in E}f_{i,(u, v)} = \sum_{(v, u) \in E}f_{v, u}^i,  ~~\forall F_i \in \mathcal{F}, \forall v \in V - \{sv_i,dv_i\} \\ 
			&\sum_{i=1}^{f}f_{i,(u, v)} \leq c_{(u, v)}, ~~~~~~~~~~~~~~~~\forall (u, v) \in E \\
			&f_{(u, v)}^i \geq 0, ~~~~~~~~~~~~~~~~~~~~~~~~~~~~~~\forall F_i \in \mathcal{F},  \forall (u, v) \in E \\
			&f_i \in \{0, 1\} ~~~~~~~~~~~~~~~~~~~~~~~~~~~~~~\forall F_i \in \mathcal{F}
		\end{aligned}
	\end{equation}
	
\newpage
	
		 \begin{algorithm}[t]
		 	\small
		 	\renewcommand{\algorithmicrequire}{\textbf{Input:}}
		 	\renewcommand\algorithmicensure {\textbf{Output:} }
		 	\caption{Random Rounding Algorithm for Formulation 1}
		 	\label{alg:randomrounding}
		 	\begin{algorithmic}[1] %Every Line has Number Label
		 		\REQUIRE ~~\\ %Input
		 		Directed Connection Graph $G(V, E)$;\\
		 		File set $F = \{F_1, ..., F_n\}$;\\
		 		Source-sink pair $(sv_i, dv_i)$ and threshold $\lambda$ associated for each $ F_i \in F$;\\
		 		Capacity $c(u, v), ~ \forall (u, v)\in E$;\\
		 		\ENSURE ~~ $OPT_{ALG}$\\ 
%		 		\STATE Divide all edge capacities by $1+\epsilon$, where $\epsilon  \in [0, 1]$, i.e., $\tilde{c_e} = \frac{c_e}{(1+\epsilon)}$;
		 		\STATE Change the last constraint to be 
		 		$0 \leq f_i \leq 1$;
		 		\STATE Then it is relaxed to an LP, solve this LP and get optimal solution $\tilde{f_i}$;
		 		\STATE With probability $\tilde{f_i}$, set $f_i = 1$, otherwise set it to $0$;
		 		\STATE Scale up the fractional flow $\tilde{f_{i,e}}$ from the LP solution on edge $e$ for commodity $i$ by $\frac{1}{\tilde{f_i}}$, i.e., $f_{i,e} = \tilde{f_{i,e}} \times \frac{1}{\tilde{f_i}}$;

		 	\end{algorithmic}
		 \end{algorithm}
	
	
	
Let $OPT_i$ be the optimal solution of the IP in Formulation 1, and let $OPT_f$ be the total amount commodities delivered by solving the linear relaxation LP of IP in Formulation 1 where the variables $f_i$ are relaxed to assume any value in $[0, 1]$. It is obvious to see that $OPT_f = \sum{\tilde{f_i}}$. Define the solution from the above algorithm as $ALG$, and total amount of commodities delivered by $ALG$ as $OPT_{ALG}$. 

We use Chernoff Bound over continuous random variables to bound the probability of achieving a fraction of the optimal solution.

\textbf{Fact 1} (Chernoff-Bound)\label{CB} . Let $X=\sum_{i=1}^{n}X_i$ be a sum of n independent random variables $X_i \in [0,1], 1\le i \le n$. Then $P(X<(1-\epsilon) \cdot E(X)) \le exp(-\epsilon^2 \cdot E(X)/2)$ holds for $0 < \epsilon < 1$.

Since we have already scaled down the flow by $s_r$, thus the variables are between 0 and 1 so that we can apply the above Chernoff Bound.

\vspace*{0.15in}
\textbf{Claim 2}. $Pr[OPT_{ALG} < (1-\epsilon) \cdot OPT_f] \le e^{-\epsilon^2 \cdot OPT_f / 2}$ 
\vspace*{0.15in}

\textbf{Proof}: For each commodity $i$, the expectation of $f_i$ is $E(f_i) = 1 \cdot \tilde{f_i} + 0 \cdot (1-\tilde{f_i}) = \tilde{f_i}$. Recall that $OPT_f = \sum{\tilde{f_i}} $, and let $OPT_{ALG} = \sum{f_i}$.

\begin{eqnarray} \label{OPTALG}
Pr[OPT_{ALG}  < (1-\epsilon) \cdot OPT_f] &=& Pr[\sum{f_i} < (1-\epsilon) \cdot OPT_f] \\ \nonumber
&\le& e^{-\epsilon^2 \cdot OPT_f / 2}  \,\,\,\,\,\,\,\,\,\,\,\,\,\,\,\,\,\,\,\,\,\,\,\,\,\,\,\,\,\,\,\,\,\,\,\,\,\,\,\,\,\,\,\,\,\,\,\,\,\,\,\,\,\,\,\,\,\,\,\,\,\,\,\,\,\,\,\,\,\,\,\,\,\,\,\,\,\,\,\,\,\,\,\,\,\QEDB \nonumber
\end{eqnarray}

 Since we also assume that all commodity can be fully fractionally delivered alone, therefore, we get:

\begin{equation} \label{OPTF}
	OPT_{f} \ge 1
\end{equation}

Since we have proved that the optimal fractional solution is an upper bound of optimal solution from ILP problem, by taking $\epsilon = 2/3$ in the Chernoff Bound, we get, 

\textbf{Theorem 3}. The probability of achieving less than 1/3 of the profit of an optimal solution is upper bounded by $e^{-2/9} \approx 0.8007$.

\textbf{Proof}.
By taking $\epsilon = 2/3$, Equation \ref{OPTALG} becomes:

\begin{eqnarray}
	Pr[OPT_{ALG}  < \frac{1}{3} \cdot OPT_f] &\le& e^{-(\frac{2}{3})^2 \cdot OPT_f / 2} \\
	&=& e^{-2 \cdot OPT_f/9}
\end{eqnarray}

By Equation \ref{OPTF}, we know that the minimum value of $OPT_f$ is 1, by taking $OPT_f = 1$, we get the upper bound of $e^{-\epsilon^2 \cdot OPT_f / 2}$. Therefore,

\begin{equation}
Pr[OPT_{ALG}  < \frac{1}{3} \cdot OPT_f] \le e^{-2/9}
\end{equation}

Since $OPT_i \le OPT_f$, thus we get, 
\begin{equation}
	Pr[OPT_{ALG}  < \frac{1}{3} \cdot OPT_i] \le e^{-2/9}
	\,\,\,\,\,\,\,\,\,\,\,\,\,\,\,\,\,\,\,\,\,\,\,\,\,\,\,\,\,\,\,\,\,\,\,\,\,\,\,\,\,\,\,\,\,\,\,\,\,\,\,\,\,\,\,\,\,\,\,\,\,\,\,\,\,\,\,\,\,\,\,\,\,\,\,\,\,\,\,\,\,\,\,\,\,\QEDB \nonumber
\end{equation}


%If $OPT_f \cdot d > c_{min}$, where $d$ is the file size. Therefore, $OPT_f > \frac{c_{min}}{d} \ge \epsilon^{-2} \cdot n|E|$.  Then, $Pr[OPT_{ALG}  \ge \frac{1-\epsilon}{1+\epsilon} \cdot OPT_f]$ is at least $1-\frac{1}{poly(|E|)}$.

%If $OPT_f \ge \alpha \cdot \ln f$, then, $Pr[OPT_{ALG}  \ge \frac{1-\epsilon}{1+\epsilon} \cdot OPT_f]$ is at least $1-\frac{1}{poly(f)}$.
%
%If $OPT_f < \alpha \cdot \ln f$, then, $OPT_{ALG} < \alpha \cdot \ln f$, even if we get 0 file, we are off from $OPT_f$ by no more than $\alpha \cdot \ln f$ additive factor. Putting it all together, with probability $1-\frac{1}{poly(f)}$, we get at least $\max\{0, (1-\epsilon)/(1+\epsilon) \cdot OPT_f - \alpha \ln f\}$.

\vspace*{0.15in}
\textbf{Claim 4} Given a single edge, choose $\epsilon'$ such that $\max \frac{\tilde{f_{i,e}}}{\tilde{f_{i,e}}} \le \epsilon'$ .The probability that ALG exceeds the edge capacity constraint by a factor of $\gamma = (1+\epsilon' \cdot \sqrt{2\log|V|\Delta_F})$ is bounded by $|V|^{-4}$
\vspace*{0.15in}

We analyze the probability that our algorithm violates capacity constraints by a certain factor, we employ Hoeffding's Inequality 

\textbf{Fact 5} (Hoeffding's Inequality)\cite{Devdatt} . Let $\{X_i\}$ be independent random variables, s.t. $X_i \in [a_i, b_i]$, then $Pr(\sum_{i}X_i - E(\sum_{i}X_i ) \ge t) \le exp(-2t^2 / \sum_{i}(b_i-a_i)^2 )$ holds.

Proof: Fix an edge $e \in E$, for commodity $i$, with probability $1-\tilde{f_i}$, the flow on edge $e$ for commodity $i$ is set to 0, i.e., $f_{i,e}=0$, with probability $\tilde{f_i}$, the flow on edge $e$ for commodity $i$ is set to $\tilde{f_{i,e}} \cdot \frac{1}{\tilde{f_i}}$. Let $\Delta_F$ be the number of files going through edge $e$. 

Then the expectation of $f_{i,e}$ is 
\begin{equation}
E(f_{i,e}) = \tilde{f_{i,e}} \cdot \frac{1}{\tilde{f_i}} \cdot  \tilde{f_i}+ 0 \cdot (1-\tilde{f_i}) = \tilde{f_{i,e}}
\end{equation}

Let $F_e$ denotes the flow on edge $e$ by $ALG$, then $F_e = \sum_{i, f_{i,e} \neq 0}{f_{i,e}}$ and the expectation of $F_e$ is 
\begin{equation} \label{expect}
E[F_e] = \sum_{i, f_{i,e} \neq 0}\tilde{f_{i,e}} \cdot \frac{1}{\tilde{f_i}} \cdot  \tilde{f_i} = \sum_{i, f_{i,e} \neq 0}\tilde{f_{i,e}}
\end{equation}

%Let $\epsilon' = \frac{\epsilon}{\sqrt{2\log|V|\Delta_F}}$, since $\tilde{f_{i,e}} \ge \frac{1}{s}$, and $\tilde{f_{i,e}} \le \tilde{c_e}$, therefore, 
%\begin{equation} \label{e3}
%	\frac{\tilde{f_{i,e}}}{\tilde{f_{i,e}}} \le s \cdot \tilde{c_e},
%\end{equation}
%since we choose $\epsilon \ge s \cdot \sqrt{2\log|V|f}$, and $f \ge \Delta_F$ thus, 
%\begin{equation}\label{e4}
%\epsilon' = \frac{\epsilon}{\sqrt{2\log|V|\Delta_F}} \ge \frac{s \cdot \sqrt{2\log|V|f}}{\sqrt{2\log|V|\Delta_F}} \ge s
%\end{equation}
%Given \ref{e3} and \ref{e4}, we can get
%\begin{equation}
%	\frac{\tilde{f_{i,e}}}{\tilde{f_{i,e}}} \le \epsilon' \cdot \tilde{c_e},
%\end{equation}


Let $t=\epsilon' \cdot \sqrt{2\log|V|\Delta_F}$, by applying Hoeffding's Inequality, we get,
	
\begin{eqnarray}
Pr[F_e - E(F_e) \ge t] &\le& exp(\frac{-2t^2}{\sum_{i}(\frac{\tilde{f_{i,e}}}{\tilde{f_{i,e}}})^2}) \\
&\le& exp(\frac{-2 \cdot \epsilon'^2 \cdot \log|V| \cdot \Delta_F}{\epsilon'^2 \cdot\Delta_F}) \\
&=& |V|^{-4}
\end{eqnarray}


Given Equation \ref{expect}, and let $\gamma = (1+\epsilon' \cdot \sqrt{2\log|V|\Delta_F})/ (1+\epsilon)$,
\begin{eqnarray}
	&&Pr[F_e - \tilde{c_e} \ge \epsilon' \cdot \sqrt{2\log|V|\Delta_F}] \\
	&=& Pr[F_e \ge (1+\epsilon' \cdot \sqrt{2\log|V|\Delta_F})] \\
	&=& Pr[F_e \ge \gamma \cdot c_e] \\
	&\le& |V|^{-4}
\end{eqnarray}

%Let $X_j = \frac{f_{j,e}}{d}$, then, $F(e) \ge (1+\epsilon) \cdot \tilde{c_e}$ iff
%
%$\sum_{j, f_{j,e} \neq 0}X_j \ge (1+\epsilon) \cdot \frac{\tilde{c_e}}{d}$.  
%Therefore, 
%$Pr[F(e) \ge (1+\epsilon) \cdot \tilde{c_e}] \le e^{-\beta(\epsilon) \cdot \tilde{
%c_e}/d}$. By scaling up the capacities by $\epsilon$, i.e., $c_e = (1+\epsilon) \cdot \tilde{c_e}$, and $\beta(\epsilon) \ge \frac{2 \epsilon^2}{4.2+\epsilon}$, and by assumption $\tilde{c_e}/d \ge \frac{4.2+\epsilon}{\epsilon^2} \cdot ln|E| $, we have  $Pr[F(e) \ge c_e] \le \frac{1}{|E|^2}$. 

There are at most $|V|^2$ edges, by applying union bound over all edges using Claim 4, we can get that the probability that ALG exceeds any of the edge capacity constraints by a factor of $\gamma = (1+\epsilon' \cdot \sqrt{2\log|V|\Delta_F})$ is upper bounded by $|V|^{-2}$.



\section{Related Work}

In the recent years, MCFP has drawn significant attention from the researcher community and has been extensively studied. In \cite{1}, the authors presented a poly-logarithmic approximation method for the Edge-Disjoint Paths with Congestion problem in an undirected graph, it is a more generalized proble than classic Edge-Disjoint Paths problem (EDP) when the congestion constant is larger than 1, the authors gave an O(poly log k)-approximation algorithm for EDPwC with congestion c = 2.  Later on in \cite{6}, the authors presented a (polylog(n), poly(log log n))-approximation with congestion 1, and in \cite{7}, the author proposed a randomized O(n3/7 poly (log n))-approximation algorithm with congestion two. 

In \cite{2}, the authors studied Maximum Node Disjoint Paths problem (MNDP) with congestion also in an undirected graph, and a polynomial time algorithm is proposed to route $\omega(OPT/poly log k)$ pairs with constant congestion. 

The MCFP problem is a generalized maximum edge-disjoint paths problem, where it does not require integral flows for the source sink pairs. MCFP is a well known APX-hard even in the case of the underlying graph is a tree, and there existing a 2-approximation for tree. In \cite{3}, the authors proposed an $O(1)$ approximation algorithm for the all-or-nothing multicommodity flow problem in planar graphs, and also proved that the integrality gap is $O(1)$.


In \cite{4}, the authors studied the multicommodity flow and cut problem in polymatroidal networks, where there are submodular capacity constraints on the edges incident to a node. The underlying graph could be either directed or undirected, by analyzing the dual of the flow relaxations via continuous extensions of the Lovász extension. 

Several other applications also involve the MCFP, such as \cite{5}, solve the MCFP to help for designing route networks for container ships, in particular, the authors studied the MCFP with transit time constraints and proved that including time constraints does not necessarily increase the computational time. 


In \cite{8}, the authors also studied the undirected EDP problem but with a moderate restriction on graph connectivity, it requires global minimum cut of G to be $\Omega(\log^5n)$ so that the algorithm can be applied to graphs with high diameters and asymptotically large minors.









\end{document}

\bibliographystyle{alpha}
\begin{thebibliography}

\bibitem{Devdatt} Devdatt P. Dubhashi, Alessandro Panconesi:
Concentration of Measure for the Analysis of Randomized Algorithms. Cambridge University Press 2009, ISBN 978-0-521-88427-3, pp. I-XIV, 1-196

\bibitem{1} A Polylogarithmic Approximation Algorithm for Edge-Disjoint Paths with Congestion 2

\bibitem{2} Poly-logarithmic Approximation for Maximum Node Disjoint Paths
with Constant Congestion

\bibitem{3} Routing in undirected graphs with constant congestion

\bibitem{4} Multicommodity Flows and Cuts in Polymatroidal Networks

\bibitem{5} The time constrained multi-commodity network flow problem and its application to liner shipping network design.

\bibitem{6} Approximation Algorithms for the Edge-Disjoint Paths Problem via
Racke Decompositions .

\bibitem{7} Breaking o(n1/2)-approximation algorithms for the edge-disjoint paths problem with congestion two

\bibitem{8} Edge Disjoint Paths in Moderately Connected Graphs

\bibitem{9} Large-treewidth graph decompositions and applications

\bibitem{10} Near-optimal hardness results and approximation algorithms for edge-disjoint paths and related problems

\bibitem{11} Approximation Algorithms for the Unsplittable Flow Problem

\bibitem{12} Chekuri, Chandra, Sanjeev Khanna, and F. Bruce Shepherd. "The all-or-nothing multicommodity flow problem." SIAM Journal on Computing 42.4 (2013): 1467-1493.

\end{thebibliography}